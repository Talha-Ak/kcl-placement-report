% Report template by Stephen Kell
% based on one by Markus Kuhn at the University of Cambridge Computer Laboratory

% Modified by Talha Abdulkuddus

\documentclass[british,12pt,a4paper]{report}

\usepackage{graphicx}
\usepackage{svg}
\graphicspath{{img/}}

\usepackage[pdfusetitle,hidelinks]{hyperref}
\usepackage{xcolor}
\hypersetup{
    colorlinks,
    linkcolor={red!50!black},
    citecolor={blue!50!black},
    urlcolor={blue!80!black}
}

\usepackage{natbib}
\usepackage{glossaries}
\loadglsentries{tex/acronyms}
\makeglossaries

\usepackage[useregional]{datetime2}
\DTMlangsetup[en-GB]{ord=raise}

\begin{document}
\title{Placement Report}
\author{Talha Abdulkuddus}

\begin{titlepage}
    \vspace*{1mm}
    \begin{center}
        \Huge Placement Report
    \end{center}
    \addvspace{25mm}
    \begin{center}
        \LARGE Author Name
    \end{center}
    \addvspace{35mm}
    \begin{center}
        \includegraphics[width=40mm]{KCL-Logo.png}
    \end{center}
    \addvspace{28mm}
    \begin{center}
        \Large Position name\\
        %Computer Laboratory\\
        Organisation  name
    \end{center}
    \addvspace{15mm}
    \begin{center}\Large dates\end{center}
\end{titlepage}

\setcounter{page}{2} % titlepage otherwise forces following page to be 1

\vfill

\section*{Originality avowal}

I verify that I am the sole author of this report, except where explicitly
stated to the contrary.

I grant permission for this report to be archived and circulated
internally within King's College London, for academic purposes
(e.g.\ instruction of future students).

\bigskip

Name: Talha Abdulkuddus

\medskip

Signature:
\IfFileExists{./img/signature.svg}
{
    \(
    \begin{array}{l}
        \includesvg[width=20mm]{signature}
    \end{array}
    \)
}
{Talha Abdulkuddus}


\medskip

Date: \today

\vfill

\section*{Abstract}

The abstract should be a concise summary of the entire document's contents.
It should not be a duplicate of the introduction, but rather,
should allow a relatively expert reader to appraise themselves quickly
of what is included in the report,
including the nature of its main content and conclusions.
It should be roughly half a page in length.

\vfill

\tableofcontents
% \listoffigures
% \listoftables

\chapter{Introduction}

\section{The company}

My year in industry was spent abroad in Turkey at ILITRON Energy and Information Technologies (known
in Turkish as ILITRON Enerji ve Bilgi Teknolojileri). The placement took place from June 2022 to May
2023, and was a hybrid of remote and on-site working.

Established in 2015, ILITRON engages in engineering research \& development, with business solutions
ranging from \acro{iot} devices to autonomous robotic systems. The company at the time had recently
shifted to developing robotic systems where both hardware and software are engineered within the
company. ILITRON also works with government research programs from Turkey and the EU, focusing on
robotic hardware and software research.

At the start of the placement, the company only consisted of 10-15 employees. The company received a
major investment during my employment, causing it to expand rapidly, with the company having
employed over 50 people by the time my placement ended.

\section{Expectations}

Prior to the placement, I was unsure of what career path I would want to pursue after attaining my
degree. The main objective I hoped to achieve within the placement was to get a clearer
understanding on the area of Computer Science I would like to work in, and whether research would be
a potential avenue of interest.

I wanted to use this opportunity to try and broaden my skill set and ability, as at the time I did
not see \acro{ai} \& \acro{ml} as career paths I would want to take. Despite being captivated by the
topics, I had frequently brushed it aside due to the (perceived) difficulty of the area.

This placement would be my first position primarily focused on programming, which I hoped would
allow me to apply theory I learnt in university, and allow me to gain real-world experience on
working in a professional environment. This would also help further my employability post-degree,
and develop skills beneficial for the remainder of my studies.

Going into the placement, I was aware of the types of projects I would be working on, including the
research role I would begin with. I had some expectation of furthering my \acro{ai} knowledge beyond
what I acquired in my university modules, however I did not expect the extent of which it ended up
being. I also expected to work with colleagues across different areas of engineering.

\section{My Role}

My role within the company varied across the duration of the placement. At the start of the
placement, my primary role was assisting in research and experimentation for a research paper in
progress. This involved working with several colleagues across the globe (both within the company
and externally), presenting data and extracting insights from results. This role was later expanded
to more general software engineering responsibilities such as web app development for an internal
project management tool, and software development for robotic arm motion.

My general skills as a software engineer was also put into use when I was tasked to help in
authoring an \acro{ai}-related project proposal for a government research funding programme along
with several other colleagues, requiring mocking up potential solutions to the goal we were trying
to achieve.

\section{Teams \& training}

This placement initially took place remotely as limited transportation options to the company
offices were available, but this later moved towards a hybrid approach of in-person and remote
working once transportation could be arranged.

During my time at ILITRON, I worked with several people across different areas of the company for
the projects I was working on. The expertise I was surrounded by ranged from electrical and embedded
engineers to \acro{ml} researchers. Throughout the placement, I was encouraged to attend external
events and conferences related to robotics and \acro{ai} with colleagues, with the aim
of learning about the current state of industry in these areas and learning of new breakthroughs.

Training consisted of being acquainted with the company's existing IT systems and processes. For my
specific role, I was also provided with e-learning courses and literature regarding the software and
\acro{ml} frameworks I was to use, as well as the foundations of several mathematical concepts the
research paper was based around, including \acro{rl} algorithms and \acro{ml} equations.

The remainder of this report will cover the key projects I participated in through the year
highlighting my contributions, and a reflection on my placement experience.


\chapter{Report structure}

This section outlines how you should go about structuring your report.

\section{Front matter}

As already illustrated in this template, the report should start with a number of standard elements:
a cover page, an `Originality Avowal', an abstract and a table of contents.

The cover page should include general information including your name, student ID, your department,
the name of the company where you were working, and the submission date.

The first page after the cover page begins with some declarations by you as the author of the
report. These must begin with the originality avowal as it appears in this template (see earlier).

The second part of the avowal gives permission for King's to archive and circulate the report for
academic purposes.
You may delete it if you are unable or unwilling to grant this permission.
In most cases we expect this will not pose a problem.
If you are concerned by confidentiality issues, see \S\ref{sec:confidentiality} below.

You must sign and date your declaration. You may use a digital image rather than a signature in ink.

Next, the abstract provides a concise summary of the entire report.
It should be about half a page long. See the template Abstract, earlier, for more guidance.

The table of contents should list the section headings of the report, with page numbers.

\section{The introduction}

Describe the company you work in.
How large is it?
What is the nature of its business?
What is its business model?
How long has it been around?
How well is it performing?
Is there anything unusual or surprising about the company?
How does it compare with what you expected going in?
What did you hope to get out of the placement year?
These are just examples of the questions you might choose to answer by way of introduction
to the company.

Next, say something about your role.
What was it, in theory and in practice? (The two often differ.)
How did it fit in with the organisation?
Who did you work with? Did this change over time?
How did it fit in with what you knew already?
Did you get any training?
If so, describe it.
If you have a lot to say, especially about training or technical background,
you should create a separate section, after the introduction, to cover this;
in that case, you can just put a brief summary here and include a forward reference.

More generally, it is good to conclude the introduction with an outline of what you will
cover in the rest of the report.
The following sub-headings are purely a suggestion for how
you might structure what follows;
feel free to vary them if you see a good reason to.

\section{Projects and tasks}

Write one or more sections describing some specific project(s) or task(s) you worked on.
Be specific; provide detail.
Remember: the readers of your report are familiar with your subject area---you can be technical,
concrete and detailed.

Avoid writing in jargon, either from your company or from the specific technology.
As a rule of thumb, you should assume your reader has a degree in your discipline
but has been out of the field for a while.
That's not because your examiners have been out of the field, but for two reasons.
Firstly, you are liable to misjudge what is common knowledge and what is obscure or ephemeral;
you should err on the side of explaining something if you're not sure.
Secondly, the report is your chance to show what you have understood.
Explaining technical matters to a (hypothetical) unfamiliar peer
is a great way to show that you have really understood them.

Describe the context you worked in, and be careful to describe your own contribution
and where it handed off to others.

Critically evaluate your work: What was good? What could have been done better?

If you worked on more than one project, this may well be several sections or subsections.

\section{Reflections}

What did you learn? Don't just make a list; evaluate!
What was the most important thing you learned, and why?
Was anything a waste of time, and if so, why?
What went well? What didn't?
What was useful from your prior studies?
In what way did you use disciplinary knowledge in your work?
What do you wish you knew at the beginning that you didn't?
What will you do differently from now on in your studies?

Create some structure (section headings/subheadings) to present this section.
You should make precise points, rather than platitudes.
Be comparative, and justify your views.
The goal here is to critically evaluate your experience,
including technical and social aspects.
It's not a place to say simply that you had a nice time.
Comment on what was good, what could have been better,
and most importantly why you make those judgements.

\section{Conclusion}

End with a conclusion section that summarises the main points
you wish the reader to take away from your report.
A good conclusion is not simply a summary of what was previously
covered, but also---the previous sections
were all leading to this!---concisely states what has been learned
from the experience being reported on.

\chapter{Qualities of a good report}

There is no fixed recipe for a good report.
This template is itself not fixed:
you are free to extend and modify the structure suggested here,
to best serve what you want to say.
However, certain qualities are common to all excellent reports.
They cover the content we have been describing,
and they generally follow the following advice.

\section{Structure, content and form}

Pay attention to spelling, grammar and readability.
Good style matters. Every sentence should contain a verb!

Strive to write in a clear and honest style.
One timeless perspective on this issue is by \citet{orwell_politics_1946}.

Do include figures (pictures, diagrams) when it helps to illustrate what you are presenting.
Number and reference figures appropriately.

Acknowledge your sources, particularly if figures (or any other material)
are sourced from elsewhere.
Use external references where appropriate,
to substantiate your statements and connect your reader with further reading.
You may use any bibliographic style (including footnotes, or references
into a separate bibliography) but make your references as precise as you reasonably can.

Consider using appendices.
Material that may be interesting and relevant but too specific for the report itself
can be presented in an appendix.
For example, if you worked on a software project,
specific details of the system may be too lengthy and low-level for the main body of the report.
If they are relevant, put them in an appendix.
However, be aware that your examiner might not read the appendices,
so make sure that all truly important information
is presented in the main body of the report.

A good report does not simply state what you did; it must explain, evaluate, reflect,
criticise, and synthesise. Make connections to what you knew before, what you think now, what you will do in future. Structure your experience, learn from it. Tell us what you learned.

\section{Common mistakes}

Reports which fail, or come close to failing, normally commit one or more of the following errors of judgement.

\subsection{Being too short or too vague}

The most common way to fail is simply to underestimate the task,
delivering a thin report without the detail necessary to demonstrate understanding.

\subsection{Naming but not explaining}

Technical work is often dense with named products and technologies,
often with abbreviations and initialisms. It is your job, as a report's author,
to make this digestible and understandable to an outside reader.
Naming things is not the same thing as explaining them.
Show that you actually know what you are talking about,
on a conceptual level not just a superficial one.

\subsection{Being uncritical}

`Critical' doesn't mean you have to be negative,
but it does mean you have to be questioning,
analytical, and offer reasoned comment.
Some reports instead read like an advertisement for the company,
or offer only a narrative of what was done but with no critical or analytical element.

\subsection{Being unreadable}

Poor grammar, careless wording and unstructured `walls of text' are all ways
to leave your reader unable to discern the merits of whatever you were trying to say.
Read back what you write, and try to take an outsider's point of view when you do so.
Check that it can be understood.

\chapter{Practicalities}

The following notes will help you get your report submitted and marked.

\section{Length}

The report has a limit of 10,000 words (excluding front matter and appendices).
It must be formatted for A4 paper, with a minimum 11pt font size and 2.5cm margins.
Remember that the ability to express yourself concisely and clearly is an important
skill, which is being assessed.
The word limit is a limit, not a target.
Most good reports are in the range 5,000--10,000 words,
but occasionally are shorter if they are tightly written.

\section{Submission and marking}

The report may be submitted as soon as the placement has ended.
All reports must be submitted no later than 31st October of the beginning of your fourth year
of study.
Earlier submission is encouraged where possible.

Reports should normally be submitted electronically via KEATS.
(See below if this presents a confidentiality problem.)

Once submitted, reports will be marked independently by the Departmental Year in Industry
tutor and one other academic.
The moderated pass/fail results will normally become available according to the college guidelines
on feedback deadlines for submitted coursework.
You can expect to receive brief written comments about your submission.

As part of the marking process, input from the placement company
(that is, from the industry supervisor)
may be taken into account if available.
The student should make the industry supervisor aware that their written feedback is solicited.
Any such industrial feedback should consist of a short report (approx.\ 1 page)
including praise and criticism of the student's contribution.
However, students should remember that the mark is given on the basis of the report,
not directly on the basis of placement work or its appraisal.

The industry placement is worth 30 credits at level 5; that is,
it is equivalent to an additional two second-year modules.
The pass/fail result obtained for the industry placement report
does not contribute to the degree average.

\section{Confidentiality}
\label{sec:confidentiality}

You may be under an obligation of confidentiality
regarding some of the details of your placement projects,
as these may concern trade secrets of your placement company.
It is in your interest to be proactive about discovering
and negotiating any confidentiality requirements.
Confidentiality issues do not justify submitting a brief or vague report,
and allowances will not be made for this during marking.
Instead, the following approaches should be taken.

The primary approach is to make the enquiries in the company
about the level of allowed disclosure and any approval process that you must follow
before submitting your report.
It will usually be a good idea to ask your industry supervisor
to approve the report before it is submitted to the Department.
Plan ahead for submission: remember that
you will have finished your placement at the point when you submit your report.
Equally, you should not hold back from making contact
(most likely by e-mail) even after you have left, if you need further guidance
on what and how you have clearance to submit.

A secondary approach, if required, is to set the technical content of your report
in a somewhat hypothetical framework.
For example, if you used a particular technology but cannot explain exactly how or
for what purposes,
you could gain credit by explaining and describing the same technology in some other,
hypothetical scenario.
It is better to take this approach only if the company will not agree to other courses of action.

We can offer companies considerable flexibility in how reports are submitted.
For example, in some (very rare) cases, the company may not allow submission of a
report electronically on KEATS. In such cases we can arrange for you to e-mail your report
to the academic coordinator directly, or (failing that) send a paper copy to the department,
with instructions for it to be returned or destroyed after marking (as appropriate).
If needed, the markers could possibly sign a non-disclosure agreement,
although this is expected to be rare.
Overall, it is better to restrict the report's distribution than to compromise its contents.

\section{Marking criteria}

The report is marked pass/fail on four criteria,
weighted approximately as shown when arriving at the overall pass/fail mark.

\paragraph{Introduction and overview (10\%)}
The report clearly and concisely describes the company/organisation in which the student worked.

\paragraph{Technical details (30\%)}
The report clearly describes the student's role and initial expectations of it.
It demonstrates a clear analysis of the problems addressed throughout the placement
(for example, as part of the individual project(s) pursued by the student).
There is clear evidence of the student transferring disciplinary knowledge
to the problem at hand and of selecting and devising appropriate strategies.

\paragraph{Critical analysis and transferable skills (40\%)}
The report reflects on all elements of the student's learning experience,
including technical and social aspects, and provides a critical analysis and suggestions
for improvement.

\paragraph{Conclusions (10\%)}
The report includes a conclusion section that
summarises the main points in the report
and provides an overall self-evaluation of the student's placement experience.

\paragraph{General scholarship (10\%)}
The writing is clear, coherent, and free of repetition.
The student is able to express a well-formed argument through the report.
The report is well structured, free of grammatical and typographical errors.
A full set of references is included, and these are appropriately structured and formatted.

\printglossaries{}
\bibliographystyle{plainnat}
\bibliography{report}

\end{document}
