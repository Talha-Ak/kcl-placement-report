\chapter{Introduction}

% This document explains how to write a good report for your year in industry.
% It is available in multiple forms:
% \href{https://keats.kcl.ac.uk/course/view.php?id=75967&section=27}{HTML embedded on Keats},
% \href{https://nms.kcl.ac.uk/stephen.kell/yini/report-keats.html}{as a freestanding HTML page},
% as part of a \href{https://nms.kcl.ac.uk/stephen.kell/yini/report.pdf}{template report in PDF format}
% and in the \href{https://nms.kcl.ac.uk/stephen.kell/yini/reportsrc.zip}{LaTeX sources for that PDF}.
% If you're using Word or something similar,
% you can produce a complete template report
% by copying the HTML in your browser and pasting it onto the end of the provided
% \href{https://nms.kcl.ac.uk/stephen.kell/yini/report-word-frontmatter.docx}{Word template of front matter}.

% The report is your chance to demonstrate the understanding that you have gained as a result of your
% time on placement. That includes technical understanding, contextual understanding of that work's
% place in the world, and understanding the relationships between your academic work and your
% placement.
%----------------------------------------------------------------------------------------------------------------------
% The report is \emph{not} expected to convey every technical aspect of the project or projects undertaken.
% Its focus should be on your learning experience. However, the report is not a formality;
% simply being able to state that you have spent the time on placement, and that your managers were
% happy, does not amount to a successful placement year. The degree credit is awarded only if you can
% produce a report that demonstrates the understanding you have gained.

% Broadly, two different types of industrial placements can be distinguished. In one scenario, a
% student is assigned to a project, which starts when the student joins the company and runs for the
% entire time of the placement. In a more typical scenario, the student will experience a range of
% projects, some to completion, but others only partially. These two types of placements will result
% in a somewhat different report structure, but with a common outline, which is described in the next
% section.

% \sloppypar{} If you have any questions after reading this document, please do not hesitate to contact the
% Departmental academic Year in Industry tutor or the Faculty administrative Year in Industry
% coordinator (contact via \texttt{global-placements@kcl.ac.uk}).

\section{The company}

My year in industry was spent abroad in Turkey at ILITRON Energy and Information Technologies
(known in Turkish as ILITRON Enerji ve Bilgi Teknolojileri).
The placement took place from June 2022 to May 2023, and was a hybrid of remote and on-site working.
Established in 2015, ILITRON engages in engineering research \& development,
with business solutions ranging from \gls*{iot} devices to autonomous robotic systems.
The company has recently shifted to developing robotic systems where
both hardware and software are engineered within the company.

At the start of the placement, the company only consisted of 10 employees.
The company received a major investment during my time there causing it to expand rapidly,
with me leaving the company having 50 employees.
There was a notable culture shift during this period which I go into detail in %[[Placement Report#Reflections]].

\section{Expectations}

Prior to the placement, I was unsure of what career path I would want to walk through after attaining my degree.
The main objective I aimed for with this placement is to get a clearer understanding
on the area of Computer Science I would like to pursue,
and whether research would be avenue I would be interested in.
Another objective I wanted to achieve is to try and broaden my current skill set and ability,
as at the time I did not see \gls*{ai} \& \gls*{ml} as viable career paths for me
due to the domain knowledge required.
This placement was also my first programming-related job,
which I hoped would allow me to gain real-world experience on working in a professional environment.

Going into the placement, I was aware of the types of projects I would be working on,
including the research role I would begin with.
I had some expectation of furthering my \gls*{ai} knowledge
beyond what I acquired in my university modules,
however I did not expect the extent of which it ended up being.
I also expected to work with colleagues across different areas of engineering.

\section{My Role}

My role within the company varied across the duration of the placement.
At the start of the placement, my primary role was assisting in researching and implementing
novel reinforcement learning applications.
This involved me working with several colleagues across the globe
(both within the company and externally), presenting results and data to uncover insights.
The knowledge would then be used to author a research paper surrounding the specific use-case
the company was researching in %([[Placement Report#Reinforcement Learning Research Paper]]).
This then expanded to further software engineering responsibilities such as web development
%([[Placement Report#Task Management Application]])
and robot software development.
%([[Placement Report#Robot Manipulation Library]])

My general skills as a software engineer was also put into use when I was tasked
to author a project proposal along with several other colleagues, %([[Placement Report#European HORIZON Proposal]])
requiring mocking up potential solutions to the goal we were trying to achieve.

The remainder of this report will cover the key projects I participated in through the year
highlighting my contributions, and a reflection on my placement experience.

