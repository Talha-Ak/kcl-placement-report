\chapter{Introduction}

\section{The company}

My year in industry was spent abroad in Turkey at ILITRON Energy and Information Technologies (known
in Turkish as ILITRON Enerji ve Bilgi Teknolojileri). The placement took place from June 2022 to May
2023, and was a hybrid of remote and on-site working.

Established in 2015, ILITRON engages in engineering research \& development, with business solutions
ranging from \acro{iot} devices to autonomous robotic systems. The company at the time had recently
shifted to developing robotic systems where both hardware and software are engineered within the
company. ILITRON also works with government research programs from Turkey and the EU, focusing on
robotic hardware and software research.

At the start of the placement, the company only consisted of 10-15 employees. The company received a
major investment during my employment, causing it to expand rapidly, with the company having
employed 50+ people by the time my placement ended. There was a notable shift in the work I did and
the company culture during this period which I go into detail in
%[[Placement Report#Reflections|Reflections]].

\section{Expectations}

Prior to the placement, I was unsure of what career path I would want to pursue after attaining my
degree. The main objective I hoped to achieve in this placement was to get a clearer understanding
on the area of Computer Science I would like to work in, and whether research would be avenue I
would be interested in.

I wanted to use this opportunity to try and broaden my skill set and ability, as at the time I did
not see \acro{ai} \& \acro{ml} as career paths I would want to take. Despite being captivated by the
topics, I had frequently brushed it aside due to the (perceived) difficulty of the area.

This placement would be my first position primarily focused on programming, which I hoped would
allow me to apply theory I learnt in university, and allow me to gain real-world experience on
working in a professional environment. This would also help further my employability post-degree,
and develop skills beneficial for the remainder of my studies.

Going into the placement, I was aware of the types of projects I would be working on, including the
research role I would begin with. I had some expectation of furthering my \acro{ai} knowledge beyond
what I acquired in my university modules, however I did not expect the extent of which it ended up
being. I also expected to work with colleagues across different areas of engineering.

\section{My Role}

My role within the company varied across the duration of the placement. At the start of the
placement, my primary role was assisting in research and experimentation for a research paper in
progress. This involved working with several colleagues across the globe (both within the company
and externally), presenting data and extracting insights from results. This role was later expanded
to more general software engineering responsibilities such as web app development for an internal
project management tool, and robotics software development for robotic arm motion.

My general skills as a software engineer was also put into use when I was tasked to help in
authoring an \acro{ai} related project proposal for a government research funding programme along
with several other colleagues, requiring mocking up potential solutions to the goal we were trying
to achieve.

\section{Teams \& training}

During my time at ILITRON, I worked with several people across different areas of the company for
the projects I was working on. The expertise I was surrounded by ranged from electrical and embedded
engineers to \acro{ml} researchers.

The company provided e-learning courses and literature regarding the software and \acro{ml}
frameworks I was to use, as well as the foundations of several mathematical concepts the research
paper was based around, including reinforcement learning algorithms and \acro{ml} equations.

The remainder of this report will cover the key projects I participated in through the year
highlighting my contributions, and a reflection on my placement experience.
