\chapter{Reflections}

This placement experience has provided me with the opportunity to put my skills into practice within
a work environment, helped me evaluate areas where I need to improve, and provided me with technical
and social opportunities I could learn from.

\section{What was learnt}

As I had worked on various projects throughout the year that varied across many areas of computer
science and software, I acquired a variety of technical skills. This included strengthening areas I
was already comfortable with, such as web app development and deployment, areas that were novel to
me, such as studying and using \acro{rl} algorithms, as well as areas I had no idea I would get to
experience.

Aside from technical skills, I was given many opportunities to improve my soft skills, most of which
are not exclusively limited to the workplace and can be applied to my general life. Prior to the
placement, task management and organization was something which I improved during university, but
was still lacking in certain areas. During the placement, I found myself developing my own personal
documentation on tasks and projects I was working on, helping me to deliver and manage my time more
effectively.

Confidence in my own abilities was something that also grew throughout the project. As my colleagues
had Master's degrees, PhDs, and many years of experience in the field, I perceived myself simply as
an inexperienced intern that should follow the lead of others. To my surprise, my opinions and
insights were listened to and acknowledged by my colleagues. Being treated an equal rather than an
intern allowed me to become more assertive in my ideas and opinions, overcoming the
\enquote{impostor syndrome} I experienced at the start.

\section{What went well (and what did not)}

During the placement, I took part in a diverse array of projects that not only broadened my horizons
but also enabled me to identify and channel my personal interests. This spanned both academic and
industrial domains, granting me the invaluable opportunity to explore a wide spectrum of challenges
and solutions.

Co-authoring a research paper was a very fulfilling accomplishment, and is a piece of work that I am
proud to show as it encompasses all the deliberating, problem-solving, analysing and testing that my
fellow researchers and I experienced. It has left me genuinely considering taking up research in the
future, an option that would have easily been disregarded prior to the placement.

Another aspect that went well was the experience of working abroad in general. Living and working in
a foreign country exposed me to an entirely new cultural and professional landscape. Working
alongside co-workers from different backgrounds and cultures was a rewarding experience that
broadened my soft skills of teamwork and collaboration.

A negative of the experience was witnessing the company culture shift as it expanded during my
placement. The company was relatively small when I started, making it easier to develop close
relationships with colleagues while having an understanding on all concurrent projects being worked
on. As the company expanded due to external investment, it became more difficult to keep track of
new hires and the projects being worked on, with the initial start-up culture becoming more
corporate. While some changes did help formalise and streamline certain informal processes, I did
not feel as closely connected to the company as I did at the start.

\section{Academic influence on work}

The knowledge and skills I picked up during university was heavily used during my placement. When
working on the research paper, my academic knowledge of machine learning effectively continued from
where university stopped at, allowing me to continue applying and developing my learning skills.
Understanding academic papers often took extensive reading and background research in other
literature to fully comprehend, something that I was exposed to at university.

Another useful skill from university was the ability to convert academic theory into code, something
that was crucial and was utilised daily. Technical knowledge regarding software development
processes and data structures helped me produce clean code that could easily be worked upon.
Participation in university group projects helped in communicating and coordinating with my
colleagues in an effective and productive manner. While not perfectly analogous to collaboration in
the workplace, it allowed me to set realistic goals and expectations on myself and others.

\section{What I wish I knew beforehand}

Surrounding myself with electrical and embedded engineers that often worked with hardware at a lower
level, I wish I had taken earlier opportunities to work on microcontrollers and similar equipment.
Working with robotic hardware in the placement was the first time I had taken such opportunity, and
that brief experience provided me with a better understanding on constraints that need to be taken
into account when writing software for these platforms.

Additionally, my earlier perception of work was a rigid set of hours, with work beginning and ending
at those hours. I wish I knew beforehand that working is not always so rigid, with my working hours
varying across the duration of my placement. While there were certainly core hours to ensure team
collaboration, I quickly realized that the nature of software engineering (and research for that
matter) allowed for a level of autonomy in managing my schedule.

\section{Changing my approach to university}

This placement has allowed me to finally identify specific areas of study that engage me the most,
and I intend to pursue them in my projects and modules during my final year. Tackling challenging
projects during my placement has provided me the confidence to approach more challenging topics and
projects in my degree. Additionally, the demands of the placement forced me to sharpen my soft
skills (notably time management and organisational skills), ensuring that I am managing my workload
efficiently and am better prepared for my studies. I am also better equipped to participate and lead
group projects, having experience on what realistic goals and expectations I can set on colleagues.
